\documentclass{article}

\title{Complexity Measures}
\author{Vladimir Feinberg}

\usepackage{nicefrac}
% latex math commands
% Vladimir Feinberg
% \vx notation for vectors from Goodfellow
% https://github.com/goodfeli/dlbook_exercises

% alphabet templates
% abcdefghijklmnopqrstuvwxyz
% ABCDEFGHIJKLMNOPQRSTUVWXYZ

% fonts, math, and layout 
\usepackage{bbm}
\usepackage{enumerate}
\usepackage{amsmath}
\usepackage{amsthm}
\usepackage{amssymb}
\usepackage{amsfonts}
\usepackage{mathrsfs}
\usepackage{mathtools}
\usepackage[all]{xy}

% include graphics with \includegraphics
\usepackage{graphicx}
\usepackage{caption}

% \nicefrac{x}{y} gives a diagonal fraction bar x/y
\usepackage{nicefrac}

% definitions
\newtheorem{theorem}{Theorem}
\theoremstyle{definition}
\newtheorem{definition}{Definition}[section]

% url
\usepackage{hyperref}
\hypersetup{
    colorlinks=true,
    linkcolor=blue,
    filecolor=magenta,      
    urlcolor=cyan,
}

% brackets, norms, cardinalities
\newcommand{\pa}[1]{ \left({#1}\right) }
\newcommand{\ha}[1]{ \left[{#1}\right] }
\newcommand{\ca}[1]{ \left\{{#1}\right\} }
\newcommand{\inner}[1]{\left\langle #1 \right\rangle}
\newcommand{\innercpy}[1]{\inner{ #1, #1 }}
\newcommand{\norm}[1]{\left\lVert #1 \right\rVert}
\newcommand{\card}[1]{\left\lvert{#1}\right\rvert}
\newcommand{\abs}[1]{\card{#1}}

% math vectors
\newcommand{\va}{\mathbf{a}}
\newcommand{\vb}{\mathbf{b}}
\newcommand{\vc}{\mathbf{c}}
\newcommand{\vd}{\mathbf{d}}
\newcommand{\ve}{\mathbf{e}}
\newcommand{\vf}{\mathbf{f}}
\newcommand{\vg}{\mathbf{g}}
\newcommand{\vh}{\mathbf{h}}
\newcommand{\vi}{\mathbf{i}}
\newcommand{\vj}{\mathbf{j}}
\newcommand{\vk}{\mathbf{k}}
\newcommand{\vl}{\mathbf{l}}
\newcommand{\vm}{\mathbf{m}}
\newcommand{\vn}{\mathbf{n}}
\newcommand{\vo}{\mathbf{o}}
\newcommand{\vp}{\mathbf{p}}
\newcommand{\vq}{\mathbf{q}}
\newcommand{\vr}{\mathbf{r}}
\newcommand{\vs}{\mathbf{s}}
\newcommand{\vt}{\mathbf{t}}
\newcommand{\vu}{\mathbf{u}}
\newcommand{\vv}{\mathbf{v}}
\newcommand{\vw}{\mathbf{w}}
\newcommand{\vx}{\mathbf{x}}
\newcommand{\vy}{\mathbf{y}}
\newcommand{\vz}{\mathbf{z}}
\newcommand{\vzero}{\mathbf{0}}
\newcommand{\vone}{\mathbf{1}} 
\newcommand{\valpha}{{\boldsymbol\alpha}}
\newcommand{\vepsilon}{{\boldsymbol\epsilon}}
\newcommand{\veta}{{\boldsymbol\eta}}
\newcommand{\vsigma}{ {\boldsymbol\sigma}}
\newcommand{\vtheta}{ {\boldsymbol\theta}}
\newcommand{\vdelta}{ {\boldsymbol\delta}}
\newcommand{\vlambda}{ {\boldsymbol\lambda}}
\newcommand{\vmu}{ {\boldsymbol\mu}}
\newcommand{\vvartheta}{ {\boldsymbol\vartheta}}
\newcommand{\vbeta}{ {\boldsymbol\beta}}
\newcommand{\vphi}{ {\boldsymbol\phi}}


% math function arrows, misc binary math ops
\newcommand{\bij}{\leftrightarrow}
\newcommand{\inj}{\rightarrowtail}
\newcommand{\sur}{\twoheadedrightarrow}
\newcommand{\eqd}{\mathrel{\overset{\Delta}{=}}}

% common math sets
\newcommand{\Z}{\mathbb{Z}}
\newcommand{\R}{\mathbb{R}}
\newcommand{\C}{\mathbb{C}}
\newcommand{\N}{\mathbb{N}}
\newcommand{\Q}{\mathbb{Q}}
\newcommand{\F}{\mathbb{F}}
\newcommand{\T}{\mathbb{T}}

% limits
\def\sumn{\sum_{n=0}^\infty}
\def\limn{\lim_{n\rightarrow\infty}}
\def\prodn{\prod_{n=0}^\infty}

% mathcal
\newcommand{\mcA}{\mathcal{A}}
\newcommand{\mcB}{\mathcal{B}}
\newcommand{\mcC}{\mathcal{C}}
\newcommand{\mcD}{\mathcal{D}}
\newcommand{\mcE}{\mathcal{E}}
\newcommand{\mcF}{\mathcal{F}}
\newcommand{\mcG}{\mathcal{G}}
\newcommand{\mcH}{\mathcal{H}}
\newcommand{\mcI}{\mathcal{I}}
\newcommand{\mcJ}{\mathcal{J}}
\newcommand{\mcK}{\mathcal{K}}
\newcommand{\mcL}{\mathcal{L}}
\newcommand{\mcM}{\mathcal{M}}
\newcommand{\mcN}{\mathcal{N}}
\newcommand{\mcO}{\mathcal{O}}
\newcommand{\mcP}{\mathcal{P}}
\newcommand{\mcQ}{\mathcal{Q}}
\newcommand{\mcR}{\mathcal{R}}
\newcommand{\mcS}{\mathcal{S}}
\newcommand{\mcT}{\mathcal{T}}
\newcommand{\mcU}{\mathcal{U}}
\newcommand{\mcV}{\mathcal{V}}
\newcommand{\mcW}{\mathcal{W}}
\newcommand{\mcX}{\mathcal{X}}
\newcommand{\mcY}{\mathcal{Y}}
\newcommand{\mcZ}{\mathcal{Z}}

% measure theory, distributions
\newcommand{\indicator}{\mathbbm{1}}
\DeclareMathOperator{\Laplace}{Laplace}
\DeclareMathOperator{\soft}{soft}
\DeclareMathOperator{\hard}{hard}
\DeclareMathOperator{\vcdim}{vc}
\DeclareMathOperator{\clo}{clo}
\DeclareMathOperator{\Poisson}{Poisson}
\DeclareMathOperator{\PoissonBinomial}{PoissonBinomial}
\DeclareMathOperator{\Exponential}{Exponential}
\DeclareMathOperator{\Multinomial}{Multinomial}
\DeclareMathOperator{\Bernoulli}{Bernoulli}
\DeclareMathOperator{\Categorical}{Categorical}
\DeclareMathOperator{\Uniform}{Uniform}
\DeclareMathOperator{\Binomial}{Binomial}
\DeclareMathOperator{\Hypergeometric}{Hypergeometric}
\DeclareMathOperator{\GammaDist}{Gamma}
\DeclareMathOperator{\NegativeBinomial}{NegativeBinomial}
\DeclareMathOperator\mathProb{\mathbb{P}}
\DeclareMathOperator\sub{sub}
\renewcommand{\d}[1]{\mathop{\mathrm{d} #1 }}
\DeclarePairedDelimiterX{\infdivx}[2]{(}{)}{ #1\;\delimsize\|\;#2 }
\newcommand{\dkl}[2]{\mathop{D_\text{KL}}\infdivx{#1}{#2}}
\newcommand{\sg}{\mathop{\mathrm{SG}}}
\newcommand{\se}{\mathop{\mathrm{SE}}}

% distributions
\makeatletter
\newcommand{\distas}[1]{\mathbin{\overset{#1}{\kern\z@\sim}}}%
\makeatother
\newcommand{\disteq}{\mathrel{\overset{\Delta}{=}}}
\newcommand{\lpeq}{\mathrel{\overset{L^2(\P)}{=}}}
\newcommand{\distiid}{\distas{\text{iid}}}
\newcommand\independent{\protect\mathpalette{\protect\independenT}{\perp}}
\def\independenT#1#2{\mathrel{\rlap{$#1#2$}\mkern2mu{#1#2}}}
\newcommand{\convdist}{\mathbin{\overset{\text{d}}{\longrightarrow}}}
\newcommand{\convarg}[1]{\mathbin{\overset{ #1 }{\longrightarrow}}}
\newcommand{\convlp}{\mathbin{\overset{L^2(\P)}{\longrightarrow}}}
\newcommand{\convlpi}{\mathbin{\overset{L^1(\P)}{\longrightarrow}}}
\newcommand{\convas}{\mathbin{\overset{\text{as}}{\longrightarrow}}}
\newcommand{\convpb}{\mathbin{\overset{\text{pb}}{\longrightarrow}}}

\renewcommand{\P}{\mathProb} % need to overwrite stupid paragraph symbol
\DeclareMathOperator\mathExp{\mathbb{E}}
\DeclareMathOperator*\mathExpUnder{\mathbb{E}}
\newcommand{\E}{\mathExp}

\newcommand{\sa}{{\(\sigma\)-algebra}}
\newcommand{\OR}{{\overline{\R}}}
\newcommand{\OX}{{\overline{X}}}
\DeclareMathOperator{\power}{{\mathcal{P}}}
\DeclareMathOperator{\var}{var}
\DeclareMathOperator{\cov}{cov}

% \set{from set}{condition} with set-builder notation
% conditional expectation is analogous
\newcommand{\set}[2]{ \left\{ #1 \,\middle|\, #2 \right\} }
\newcommand{\CE}[2]{ \mathExp\left[ #1 \,\middle|\, #2 \right] }

% linear-algebra related
\newcommand{\rip}{\operatorname{RIP}}
\DeclareMathOperator{\svd}{svd}
\newcommand{\frob}[1]{\norm{#1}_{\mathrm{F}}}
\newcommand{\mat}[1]{\begin{pmatrix} #1 \end{pmatrix}}
\newcommand{\detmat}[1]{\begin{vmatrix} #1 \end{vmatrix}}
\DeclareMathOperator{\spanb}{span}
\DeclareMathOperator{\conv}{conv} % convex hull
\DeclareMathOperator{\cone}{cone}
\DeclareMathOperator{\vectorize}{vec}
\DeclareMathOperator{\matricize}{mat}
\DeclareMathOperator{\adj}{adj}
\DeclareMathOperator{\diag}{diag}
\DeclareMathOperator{\tr}{tr}
\DeclareMathOperator{\rank}{rank}
\DeclareMathOperator*{\argmin}{argmin}
\DeclareMathOperator*{\argmax}{argmax}
\DeclareMathOperator*{\proj}{proj}

% complex analysis
\DeclareMathOperator{\MyRe}{Re}
\DeclareMathOperator{\MyIm}{Im}
\DeclareMathOperator{\image}{image}
\DeclareMathOperator{\supp}{supp}

% typical numerical operators
\DeclareMathOperator{\sgn}{sgn}

% graphs
\DeclareMathOperator{\diam}{diam}

% constants
\renewcommand{\d}[1]{\mathop{\mathrm{d} #1 }}
\newcommand{\e}{\mathrm{e}}
\renewcommand{\i}{\imath}

% \bigtimes: large indexed cross product
\makeatletter
\DeclareFontFamily{U}  {MnSymbolF}{}
\DeclareSymbolFont{symbolsMN}{U}{MnSymbolF}{m}{n}
\SetSymbolFont{symbolsMN}{bold}{U}{MnSymbolF}{b}{n}
\DeclareFontShape{U}{MnSymbolF}{m}{n}{
    <-6>  MnSymbolF5
   <6-7>  MnSymbolF6
   <7-8>  MnSymbolF7
   <8-9>  MnSymbolF8
   <9-10> MnSymbolF9
  <10-12> MnSymbolF10
  <12->   MnSymbolF12}{}
\DeclareFontShape{U}{MnSymbolF}{b}{n}{
    <-6>  MnSymbolF-Bold5
   <6-7>  MnSymbolF-Bold6
   <7-8>  MnSymbolF-Bold7
   <8-9>  MnSymbolF-Bold8
   <9-10> MnSymbolF-Bold9
  <10-12> MnSymbolF-Bold10
  <12->   MnSymbolF-Bold12}{}
\DeclareMathSymbol{\tbigtimes}{\mathop}{symbolsMN}{2}
\newcommand*{\bigtimes}{%
  \DOTSB
  \tbigtimes
  \slimits@ 
}
\makeatother

% category theory arguments
% See https://tex.stackexchange.com/questions/356873
\newcommand{\catfst}{{-}}
\newcommand{\catsnd}{{=}}
\newcommand{\cattrd}{{\equiv}}
\DeclareMathOperator{\Id}{Id}
\newcommand{\opcat}[1]{{#1}^{\text{op}}}
\DeclareMathOperator{\Hom}{hom}
\DeclareMathOperator{\Ob}{ob}
\DeclareMathOperator{\El}{el}
\DeclareMathOperator{\colim}{colim}
\DeclareMathOperator{\Sym}{Sym}
% restriction of a function to a domain
\newcommand\restr[2]{{% we make the whole thing an ordinary symbol
  \left.\kern-\nulldelimiterspace % automatically resize the bar with \right
  #1 % the function
  \vphantom{\big|} % pretend it's a little taller at normal size
  \right|_{#2} % this is the delimiter
}}

\newtheorem{lemma}{Lemma}


\begin{document}

\maketitle

Complexity measures evaluate the expressiveness of a hypothesis class; they are useful to the extent with which they relate sample and generalization error.

Notation. TODO move to common place in repo. Big-$O$ notation (and $\Theta,\Omega$) is used in the standard sense for single variables: $f=O(g)$ if there exists a $c$ such that $f(x)\le cg(x)$ for all $x$ sufficiently large. In multiple variables, big-$O$ and similarly $\Theta,\Omega$ require $f(\Tx)\le cg(\Tx)$ for all $\norm{x}_\infty$ sufficiently large. To capture the essence of asymptotics, a tilde will crudely capture asymptotic behavior up to logarithmic factors: $f=\tilde{O}(g)$ if $f=O(g\log^k g)$ for some $k\in\N$.

Abbreviations. TODO move to common place.

\section{Setup}

We suppose that our data comes in the form of ordered pairs from $\mcX\times \mcY$. Samples follow a particular distribution $(x, y)\sim D$. A hypothesis class $\mcH$ is set of functions $\mcX\rightarrow\mcY$.

A common approach to supervised learning is empirical risk minimization, where $m$ iid samples from $D$, $S$, are used to find the $h\in\mcH$ minimizing a specified loss $\ell:\mcY^2\rightarrow\R$ over this set. Complexity measures then let us quanify exactly how much loss we can expect when sampling from $D$ again.

Thus, it is useful to find bounds on $\varepsilon$, the difference between the generalization loss $\E\left[\ell\pa{h(x), y)}\right]$, where $(x,y)\sim D$, and sample loss, where the loss is the expectation before taken for $(x,y)$ is uniform over $S$.

Let the gap between the generalization and sample error be $\varepsilon$.

\section{Complexity Measures}

Rademacher complexity. TODO. relates complexity through noise correlation. equivalently, gives a consistent view of ``error'' through a ``uniform'' noise model.

The growth function. TODO.

VC-dimension. TODO.

\section{Overview of Results}

Proofs can be found in a cogent write-up by Prof. Beckage from the University of Kansas,\footnote{\url{http://ittc.ku.edu/~beckage/ml800/VC_dim.pdf}} copied into this repository.\footnote{\url{https://github.com/vlad17/shallow-ml-notes/raw/7c3db7b7c924fbd2ae2891d4ecfef84e5647dcc8/computational-learning/complexity-measures-beckage.pdf}} Recall our generalization gap definition. For a fixed $h\in \mcH$:
$$
\varepsilon= \E\left[\ell\pa{h(x), y)}|(x,y)\sim D\right]-\E\left[\ell\pa{h(x), y)}|(x,y)\sim \Uniform(S)\right]
$$

\subsection{VC Generalization Bounds}

Upper bound. If $d$ is the VC-dimension of $\mcH$, then for any $D$ wp $1-\delta$:
$$
\varepsilon\le \tilde{O}\pa{\sqrt{\frac{d-\log \delta}{m}}}
$$
The above inequality is random since it depends on $S$, the $D^m$-valued rv. TODO. find source removing tilde?

Agnostic lower bound. We may find a $D$ such that with a fixed nonzero probability (a nonneglible set of candidate samples $S$), the following holds:
$$
\varepsilon\ge \Omega\pa{\sqrt{\frac{d}{m}}}
$$

The above implies that in the common case of agnostic hypothesis learning, where we do not know distribution $D$, VC-dimension is, \textit{up to logarithmic factors, asymptotically efficient} in quantifying the generalization gap.

Realizability. Suppose $D$ is realizable wrt $\mcH$, so that almost surely there exists an $f\in\mcH$ such that for any $(x,y)$ sampled from $D$, $f(x)=y$. Then all statements above hold but with $\sqrt{\varepsilon}$ instead of $\varepsilon}$.


\subsection{Rademacher bounds}

With $R_m$ either the empirical or expected Rademacher complexity over the sample for a given $h,\ell$ we have again wp $1-\delta$:
$$
\varepsilon\le 2R_m+O\pa{\frac{\log\nicefrac{1}{\delta}}{m}}
$$
$R_m$ may be NP-hard to compute, depending on $\mcH$. This tells us Rademacher complexity could only be a useful imporvement over VC-bounds, asymptotically, if we have an efficient approximation for the empirical Rademacher complexity or some knowledge of $D$ as required to compote the true Rademacher complexity.

TODO. NP-hardness? More computational learning theory.

TODO. Rademacher and Gaussian Complexities: Risk Bounds and Structural Results by Bartlett and Mendelson.
\end{document}