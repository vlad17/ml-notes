\documentclass{article}

\title{The It\^{o} Formula and the Martingale Representation Theorem}
\author{Vladimir Feinberg}

\usepackage{fullpage}

% latex math commands
% Vladimir Feinberg
% \vx notation for vectors from Goodfellow
% https://github.com/goodfeli/dlbook_exercises

% alphabet templates
% abcdefghijklmnopqrstuvwxyz
% ABCDEFGHIJKLMNOPQRSTUVWXYZ

% fonts, math, and layout 
\usepackage{bbm}
\usepackage{enumerate}
\usepackage{amsmath}
\usepackage{amsthm}
\usepackage{amssymb}
\usepackage{amsfonts}
\usepackage{mathrsfs}
\usepackage{mathtools}
\usepackage[all]{xy}

% include graphics with \includegraphics
\usepackage{graphicx}
\usepackage{caption}

% \nicefrac{x}{y} gives a diagonal fraction bar x/y
\usepackage{nicefrac}

% definitions
\newtheorem{theorem}{Theorem}
\theoremstyle{definition}
\newtheorem{definition}{Definition}[section]

% url
\usepackage{hyperref}
\hypersetup{
    colorlinks=true,
    linkcolor=blue,
    filecolor=magenta,      
    urlcolor=cyan,
}

% brackets, norms, cardinalities
\newcommand{\pa}[1]{ \left({#1}\right) }
\newcommand{\ha}[1]{ \left[{#1}\right] }
\newcommand{\ca}[1]{ \left\{{#1}\right\} }
\newcommand{\inner}[1]{\left\langle #1 \right\rangle}
\newcommand{\innercpy}[1]{\inner{ #1, #1 }}
\newcommand{\norm}[1]{\left\lVert #1 \right\rVert}
\newcommand{\card}[1]{\left\lvert{#1}\right\rvert}
\newcommand{\abs}[1]{\card{#1}}

% math vectors
\newcommand{\va}{\mathbf{a}}
\newcommand{\vb}{\mathbf{b}}
\newcommand{\vc}{\mathbf{c}}
\newcommand{\vd}{\mathbf{d}}
\newcommand{\ve}{\mathbf{e}}
\newcommand{\vf}{\mathbf{f}}
\newcommand{\vg}{\mathbf{g}}
\newcommand{\vh}{\mathbf{h}}
\newcommand{\vi}{\mathbf{i}}
\newcommand{\vj}{\mathbf{j}}
\newcommand{\vk}{\mathbf{k}}
\newcommand{\vl}{\mathbf{l}}
\newcommand{\vm}{\mathbf{m}}
\newcommand{\vn}{\mathbf{n}}
\newcommand{\vo}{\mathbf{o}}
\newcommand{\vp}{\mathbf{p}}
\newcommand{\vq}{\mathbf{q}}
\newcommand{\vr}{\mathbf{r}}
\newcommand{\vs}{\mathbf{s}}
\newcommand{\vt}{\mathbf{t}}
\newcommand{\vu}{\mathbf{u}}
\newcommand{\vv}{\mathbf{v}}
\newcommand{\vw}{\mathbf{w}}
\newcommand{\vx}{\mathbf{x}}
\newcommand{\vy}{\mathbf{y}}
\newcommand{\vz}{\mathbf{z}}
\newcommand{\vzero}{\mathbf{0}}
\newcommand{\vone}{\mathbf{1}} 
\newcommand{\valpha}{{\boldsymbol\alpha}}
\newcommand{\vepsilon}{{\boldsymbol\epsilon}}
\newcommand{\veta}{{\boldsymbol\eta}}
\newcommand{\vsigma}{ {\boldsymbol\sigma}}
\newcommand{\vtheta}{ {\boldsymbol\theta}}
\newcommand{\vdelta}{ {\boldsymbol\delta}}
\newcommand{\vlambda}{ {\boldsymbol\lambda}}
\newcommand{\vmu}{ {\boldsymbol\mu}}
\newcommand{\vvartheta}{ {\boldsymbol\vartheta}}
\newcommand{\vbeta}{ {\boldsymbol\beta}}
\newcommand{\vphi}{ {\boldsymbol\phi}}


% math function arrows, misc binary math ops
\newcommand{\bij}{\leftrightarrow}
\newcommand{\inj}{\rightarrowtail}
\newcommand{\sur}{\twoheadedrightarrow}
\newcommand{\eqd}{\mathrel{\overset{\Delta}{=}}}

% common math sets
\newcommand{\Z}{\mathbb{Z}}
\newcommand{\R}{\mathbb{R}}
\newcommand{\C}{\mathbb{C}}
\newcommand{\N}{\mathbb{N}}
\newcommand{\Q}{\mathbb{Q}}
\newcommand{\F}{\mathbb{F}}
\newcommand{\T}{\mathbb{T}}

% limits
\def\sumn{\sum_{n=0}^\infty}
\def\limn{\lim_{n\rightarrow\infty}}
\def\prodn{\prod_{n=0}^\infty}

% mathcal
\newcommand{\mcA}{\mathcal{A}}
\newcommand{\mcB}{\mathcal{B}}
\newcommand{\mcC}{\mathcal{C}}
\newcommand{\mcD}{\mathcal{D}}
\newcommand{\mcE}{\mathcal{E}}
\newcommand{\mcF}{\mathcal{F}}
\newcommand{\mcG}{\mathcal{G}}
\newcommand{\mcH}{\mathcal{H}}
\newcommand{\mcI}{\mathcal{I}}
\newcommand{\mcJ}{\mathcal{J}}
\newcommand{\mcK}{\mathcal{K}}
\newcommand{\mcL}{\mathcal{L}}
\newcommand{\mcM}{\mathcal{M}}
\newcommand{\mcN}{\mathcal{N}}
\newcommand{\mcO}{\mathcal{O}}
\newcommand{\mcP}{\mathcal{P}}
\newcommand{\mcQ}{\mathcal{Q}}
\newcommand{\mcR}{\mathcal{R}}
\newcommand{\mcS}{\mathcal{S}}
\newcommand{\mcT}{\mathcal{T}}
\newcommand{\mcU}{\mathcal{U}}
\newcommand{\mcV}{\mathcal{V}}
\newcommand{\mcW}{\mathcal{W}}
\newcommand{\mcX}{\mathcal{X}}
\newcommand{\mcY}{\mathcal{Y}}
\newcommand{\mcZ}{\mathcal{Z}}

% measure theory, distributions
\newcommand{\indicator}{\mathbbm{1}}
\DeclareMathOperator{\Laplace}{Laplace}
\DeclareMathOperator{\soft}{soft}
\DeclareMathOperator{\hard}{hard}
\DeclareMathOperator{\vcdim}{vc}
\DeclareMathOperator{\clo}{clo}
\DeclareMathOperator{\Poisson}{Poisson}
\DeclareMathOperator{\PoissonBinomial}{PoissonBinomial}
\DeclareMathOperator{\Exponential}{Exponential}
\DeclareMathOperator{\Multinomial}{Multinomial}
\DeclareMathOperator{\Bernoulli}{Bernoulli}
\DeclareMathOperator{\Categorical}{Categorical}
\DeclareMathOperator{\Uniform}{Uniform}
\DeclareMathOperator{\Binomial}{Binomial}
\DeclareMathOperator{\Hypergeometric}{Hypergeometric}
\DeclareMathOperator{\GammaDist}{Gamma}
\DeclareMathOperator{\NegativeBinomial}{NegativeBinomial}
\DeclareMathOperator\mathProb{\mathbb{P}}
\DeclareMathOperator\sub{sub}
\renewcommand{\d}[1]{\mathop{\mathrm{d} #1 }}
\DeclarePairedDelimiterX{\infdivx}[2]{(}{)}{ #1\;\delimsize\|\;#2 }
\newcommand{\dkl}[2]{\mathop{D_\text{KL}}\infdivx{#1}{#2}}
\newcommand{\sg}{\mathop{\mathrm{SG}}}
\newcommand{\se}{\mathop{\mathrm{SE}}}

% distributions
\makeatletter
\newcommand{\distas}[1]{\mathbin{\overset{#1}{\kern\z@\sim}}}%
\makeatother
\newcommand{\disteq}{\overset{d}{=}}
\newcommand{\distiid}{\distas{\text{iid}}}
\newcommand\independent{\protect\mathpalette{\protect\independenT}{\perp}}
\def\independenT#1#2{\mathrel{\rlap{$#1#2$}\mkern2mu{#1#2}}}
\newcommand{\convdist}{\mathbin{\overset{\text{d}}{\longrightarrow}}}
\newcommand{\convarg}[1]{\mathbin{\overset{ #1 }{\longrightarrow}}}
\newcommand{\convas}{\mathbin{\overset{\text{as}}{\longrightarrow}}}
\newcommand{\convpb}{\mathbin{\overset{\text{pb}}{\longrightarrow}}}

\renewcommand{\P}{\mathProb} % need to overwrite stupid paragraph symbol
\DeclareMathOperator\mathExp{\mathbb{E}}
\DeclareMathOperator*\mathExpUnder{\mathbb{E}}
\newcommand{\E}{\mathExp}

\newcommand{\sa}{{\(\sigma\)-algebra}}
\newcommand{\OR}{{\overline{\R}}}
\newcommand{\OX}{{\overline{X}}}
\DeclareMathOperator{\power}{{\mathcal{P}}}
\DeclareMathOperator{\var}{var}
\DeclareMathOperator{\cov}{cov}

% \set{from set}{condition} with set-builder notation
% conditional expectation is analogous
\newcommand{\set}[2]{ \left\{ #1 \,\middle|\, #2 \right\} }
\newcommand{\CE}[2]{ \mathExp\left[ #1 \,\middle|\, #2 \right] }

% linear-algebra related
\newcommand{\rip}{\operatorname{RIP}}
\DeclareMathOperator{\svd}{svd}
\newcommand{\frob}[1]{\norm{#1}_{\mathrm{F}}}
\newcommand{\mat}[1]{\begin{pmatrix} #1 \end{pmatrix}}
\newcommand{\detmat}[1]{\begin{vmatrix} #1 \end{vmatrix}}
\DeclareMathOperator{\spanb}{span}
\DeclareMathOperator{\conv}{conv} % convex hull
\DeclareMathOperator{\cone}{cone}
\DeclareMathOperator{\vectorize}{vec}
\DeclareMathOperator{\matricize}{mat}
\DeclareMathOperator{\adj}{adj}
\DeclareMathOperator{\diag}{diag}
\DeclareMathOperator{\tr}{tr}
\DeclareMathOperator{\rank}{rank}
\DeclareMathOperator*{\argmin}{argmin}
\DeclareMathOperator*{\argmax}{argmax}
\DeclareMathOperator*{\proj}{proj}

% complex analysis
\DeclareMathOperator{\MyRe}{Re}
\DeclareMathOperator{\MyIm}{Im}
\DeclareMathOperator{\image}{image}
\DeclareMathOperator{\supp}{supp}

% typical numerical operators
\DeclareMathOperator{\sgn}{sgn}

% graphs
\DeclareMathOperator{\diam}{diam}

% constants
\renewcommand{\d}[1]{\mathop{\mathrm{d} #1 }}
\newcommand{\e}{\mathrm{e}}
\renewcommand{\i}{\imath}

% \bigtimes: large indexed cross product
\makeatletter
\DeclareFontFamily{U}  {MnSymbolF}{}
\DeclareSymbolFont{symbolsMN}{U}{MnSymbolF}{m}{n}
\SetSymbolFont{symbolsMN}{bold}{U}{MnSymbolF}{b}{n}
\DeclareFontShape{U}{MnSymbolF}{m}{n}{
    <-6>  MnSymbolF5
   <6-7>  MnSymbolF6
   <7-8>  MnSymbolF7
   <8-9>  MnSymbolF8
   <9-10> MnSymbolF9
  <10-12> MnSymbolF10
  <12->   MnSymbolF12}{}
\DeclareFontShape{U}{MnSymbolF}{b}{n}{
    <-6>  MnSymbolF-Bold5
   <6-7>  MnSymbolF-Bold6
   <7-8>  MnSymbolF-Bold7
   <8-9>  MnSymbolF-Bold8
   <9-10> MnSymbolF-Bold9
  <10-12> MnSymbolF-Bold10
  <12->   MnSymbolF-Bold12}{}
\DeclareMathSymbol{\tbigtimes}{\mathop}{symbolsMN}{2}
\newcommand*{\bigtimes}{%
  \DOTSB
  \tbigtimes
  \slimits@ 
}
\makeatother

% category theory arguments
% See https://tex.stackexchange.com/questions/356873
\newcommand{\catfst}{{-}}
\newcommand{\catsnd}{{=}}
\newcommand{\cattrd}{{\equiv}}
\DeclareMathOperator{\Id}{Id}
\newcommand{\opcat}[1]{{#1}^{\text{op}}}
\DeclareMathOperator{\Hom}{hom}
\DeclareMathOperator{\Ob}{ob}
\DeclareMathOperator{\El}{el}
\DeclareMathOperator{\colim}{colim}
\DeclareMathOperator{\Sym}{Sym}
% restriction of a function to a domain
\newcommand\restr[2]{{% we make the whole thing an ordinary symbol
  \left.\kern-\nulldelimiterspace % automatically resize the bar with \right
  #1 % the function
  \vphantom{\big|} % pretend it's a little taller at normal size
  \right|_{#2} % this is the delimiter
}}

\newtheorem{lemma}{Lemma}


\begin{document}

\maketitle

\section{The 1-dimensional It\^{o} Formula}

\begin{definition}[It\^{o} Process]
  Consider the probability space \((\Omega, \mcH,\P)\) where \(\mcH=\mcH_\infty\) for a filtration \(\ca{\mcH_t}\) for which the Brownian motion \(B_t\) is adapted. Let \(u\) be an \(H_t\) adapted as well, and \(L^1([0,T])\) almost surely in its first argument, and \(v\in\mcW_\mcH\), where \(\mcW_\mcH\) is defined in the previous chapter as a relaxation of \(\mcV_\mcH\) to almost surely \(L^2([0,T])\) processes for all \(T\ge 0\).

  Then an It\^{o} process is a process \(X_t\) satisfying
  \[
    \d{X_t}=u \d{t}+v\d{B_t}\,\,,
  \]
  which is shorthand for \(X_T(\omega)=X_0+\int_0^T u(t,\omega)\d{t}+\int_0^T v(t, \omega)\d{B_t}\) holding almost surely for all \(T\ge 0\).
\end{definition}

In certain cases, we can find the functional form of an It\^{o} process through the analogue of the chain rule.

\begin{theorem}[The 1-dimensional It\^{o} Formula]
  Let \(X_t\) be an It\^{o} process satisfying \(\d{X_t}=u \d{t}+v\d{B_t}\) and suppose \(g\in\mcC^2(\R_+\times\R)\). Then \(Y_t=g(t, X_t)\) is an It\^{o} process satisfying
  \[
   \d{Y_t}=\partial_1g(t, X_t)\d{t}+\partial_2g(t, X_t)\d{X_t}+\frac{1}{2}\partial_2^2g(t, X_t)\pa{\d{X_t}}^2 \,\,,
  \]
  where \((\d{X_t})^2\) is provocative sugar for \(\d{t}\).
\end{theorem}
\begin{proof}
  Letting \(V_s=v(s, \omega)\) and \(U_s=u(s, \omega)\) and desugaring, we must verify that \(g(t, X_t)-g(0, X_0)\) equals
  \[
   I(g, t, X)=\int_0^t\pa{\partial_1g(s,X_s)+U_s\partial_2g(s, X_s)+\frac{1}{2}V_s^2\partial_2^2g(s, X_s)}\d{s}+\int_0^tV_s\partial_2g(s,X_s)\d{B_s}\,\,.
  \]

  Before jumping into verification, we must first simplify the problem to work with bounded \(g\) and elementary \(u,v\). To argue without loss of generality, we first perform our reductions.

  \textbf{It suffices to consider bounded \(g\).} Define the stopping time \(\tau_n=\inf\set{s>0}{\abs{X_s}\ge n}\). Define \(g_n\) to equal to \(g\), but be supported on only \([0,n]\times[-n,n]\) and \(\partial_1g,\partial_2^2g\); by continuity and compact support each \(g_n\) and its derivatives are bounded.

  Defining \(\int_0^{t\land \tau_n}\cdots\) to be \(\int_0^{t}1\ca{s\le \tau_n}\cdots\), it's clear that \(I(g_n, t, X)=I(g,t\land \tau_n, X)\). For fixed \(t\), as \(X_t\) is \(L^2(\P\times [0,t])\)-integrable, \(\P\ca{\tau_n>t}\rightarrow 1\), which implies that \(I(g_n,t, X)=I(g, t\land \tau_n, X)\convas I(g, t, X)\). Since \(g_n\convas g\) as well, we can lift a proof of the It\^{o} formula from bounded \(g\) to unbounded.

\textbf{It suffices to consider elementary \(u,v\)}.
Furthermore, we may assume \(u,v\) are elementary functions; by integrability and adaptedness (and hence by Monotone Class Theorem) for \(u\) and It\^{o}'s construction elementary \(u_n,v_n\) exist with \(\d{X_t^{(n)}}=u_n\d{t}+v_n\d{B_t}\) for which \(X_t^{(n)}\convarg{L^1(\P)} X_t\), implying almost sure convergence along some subsequence. Then by continuous mapping theorem we have \(g(t, X_t^{(n)})\convas g(t, X_t)\). By boundedness of \(g\) each integral integrand of \(I(g, t, X^{(n)})\) is itself bounded, so by BCT \(I(g, t, X^{(n)})\convarg{L^1(\P)}I(g, t, X)\), and therefore through \(I(g, t, X^{(n)})=g(t, X_t^{(n)})\) we have almost sure equality of the limits.

  \textbf{With these amenities out of the way}, by application of Taylor's theorem at each \(g_j=g(t_j,X_{t_j})\), the difference \(g(t, X_t)-g(0, X_0)\) is given by

  \[
    \sum_j\Delta_jg=\sum_j\partial_1g_j\Delta_jt+\partial_2g_j\Delta_jX+\frac{1}{2}\partial_1^2 g_j\pa{\Delta_jt}^2+\frac{1}{2}\partial_2^2g_j\pa{\Delta_jX}^2+\partial_1\partial_2g_j\Delta_jt\Delta_jX+R_j\,\,,
  \]
  with the remainder \(R_j=o\pa{\pa{\Delta_jt}^2+\pa{\Delta_jX}^2}\).

  Under the limit of shrinking partitions, \(\sum_j\partial_1g_j\Delta_jt\rightarrow\int_0^t\partial_1g(s,X_s)\d{s}\), holding almost surely and then by a limit interchange (relying on \(g\)'s boundedness) in \(L^1(\P)\) as well.

  Similarly, we have \(\sum_j\partial_2g_j\Delta_jX\rightarrow\int_0^t\partial_2g(s,X_s)\d{B_s}\) in \(L^2(\P)\), since \(g\in \mcV(0,t)\). Next, we leverage the fact that \(u,v\) are elementary such that for sufficiently fine grids, there exist constants \(u_j,v_j\) such that
  \[
    \pa{\Delta_j X}^2=u_j^2\pa{\Delta_jt}^2+2u_jv_j\Delta_jt\Delta_jX+v_j^2\pa{\Delta_jB}^2\,\,.
  \]

  In the original sum, for some bounded \(c_j\), we find that terms \(\sum_jc_j\Delta_jt\Delta_jX,\sum_jc_j\pa{\Delta_jt}^2\) vanish in \(L^2(\P)\), the former's second moment being \(O(n^{-3})\) and the latter's being \(O(n^{-4})\).

  This leaves the conspicuous term \(\frac{1}{2}\sum_j\partial_2^2g_j\pa{\Delta_jB}^2\), which we may show tends to \(\int_0^t\partial_2^2g(s, X_s)V_s^2\d{s}\) in \(L^2(\P)\) by triangle inequality and comparing to the analogous sequence with \(\Delta_jt\) instead. A calculation involving normal moments and leveraging independent increments finds that indeed
  \[\E\ha{\pa{\frac{1}{2}\sum_j\partial_2^2g_j\pa{\pa{\Delta_jB}^2-\Delta_jt}}^2}\rightarrow 0\,\,.\]

  Since every term converges to the desired parts of our integral sum in \(L^2(\P)\) for some sequence, certainly we can choose a subsequence that converges almost surely. Since the discrete differences always equal \(g(t, X_t)-g(0, X_0)=\sum_j\Delta_jg\), the It\^{o} formula equality holds in the limit.
    \end{proof}

    \begin{theorem}[Stochastic Integration by Parts]
      If \(f(s,\omega)=f(s)\) is deterministic and of bounded variation in \([0,t]\) such that the \href{https://en.wikipedia.org/wiki/Riemann%E2%80%93Stieltjes_integral}{Riemann-Stieltjes integral} is well-defined, then
        \[
          \int_0^t f(s)\d{B_s}=f(t)B_t-\int_0^tB_s\d{f_s}
          \]
        \end{theorem}
        \begin{proof}
          In the case of \(f\in\mcC^2\), this follows from the It\^{o} formula. But for the more general class, a direct proof is needed. For a sequence of shrinking partitions of \([0, t]\),
          \[
            f(t)B_t=\sum_j\Delta_j(fB)=\sum_jB_{t_{j+1}}\Delta_j f+\sum_jf_{t_j}\Delta_j B\convas \int_0^tB_s\d{f_s}+\int_0^t f(s)\d{B_s}\,\,,
          \]
          which is apparent after adding and subtracting the term \(f_{t_j}B_{t_{j+1}}\) within \(\Delta_j(fB)=f_{t_{j+1}}B_{t_{j+1}}-f_{t_{j}}B_{t_{j}}\).
          \end{proof}
          
\section{The Multi-dimensional It\^{o} Formula}

For vector-valued \(u:(\R_+,\R^n)\rightarrow\R^n\) and matrix-valued \(v:(\R_+,\R^n)\rightarrow\R^{m\times n}\), such that \(\d{X_t}=u\d{t}+v\d{B_t^{(m)}}\) has well-defined It\^{o} process components, for a conformal \(g\in\mcC^2\), an analogous multi-dimensional It\^{o} formula may be derived in the same manner as above via Taylor's theorem.

\section{The Martinagle Representation Theorem}

Note that this theorem has a similar multi-dimensional extension.

\begin{theorem}[Martingale Representation Theorem]
  Suppose \(M_t\) is a square-integrable martinagle with respect to the filtration generated by Brownian motion. Then there exists a stochastic process \(g(s,\omega)\in \mcV(0, t)\) such that
  \[
    M_t=\E\ha{M_0}+\int_0^tg(s,\cdot)\d{B_s}\,\,,
  \]
  holding almost surely for all \(t\ge 0\). 
\end{theorem}
\begin{proof}
  The proof proceeds in several steps. First; for any \(T>0\) the set random variables defined by evaluating analytic functions with compact support on \(R^n\) on any finite tuple of Brownian motions \((B_{t_1},\cdots,B_{t_n})\) is dense in \(L^2(\mcF_T,\P)\), the set of real-valued square-integrable rvs measurable with respect to \(\mcF_T\).

  Second; the linear span of random variables of type \(\exp\pa{\int_0^Th(t)\d{B_t}-\frac{1}{2}\int_0^Th^2(t)\d{t}}\) for all (deterministic) \(h\in L^2([0,T])\) is also dense in \(L^2(\mcF_T, \P)\) by virtue of spanning the set described above over analytic functions with compact support.

  By the density of the class described above, it suffices to consider only random variables of the aforementioned form. For any such \(Y_T\) defined by some \(h\), as the final representation equality holds under linear transformation. Since \(Y_t=\exp(X_t)\), where \(\d{X_t}=-\frac{1}{2}h^2(s)\d{s}+h(s)\d{B_s}\). Applying It\^{o}'s formula yields an equality exactly of the desired form \(Y_T=1 + \int_0^tg(s,\cdot)\d{B_s}\). In turn, for any square-integrable, \(\mcF_T\)-measurable random variable, such as \(M_T\), by linearity we may write it as \(\E[M_T]+\int_0^Tg_{T}(s,\cdot)\d{B_s}\).

  This holds for all \(T\), and \(\E\ha{M_T}=\E \ha{M_0}\) uniformly, but we must show that the \(g_T\) values are the same across \(T\) values. Luckily, this holds due to Exercise 15 of Chapter 3.
  \end{proof}


\section{Exercises}

\subsection{}

Exercise 4.9 is completed inline in the proof of It\^{o}'s formula provided above.

\end{document}