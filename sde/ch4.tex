\documentclass{article}

\title{The It\^{o} Formula and the Martingale Representation Theorem}
\author{Vladimir Feinberg}

\usepackage{fullpage}

% latex math commands
% Vladimir Feinberg
% \vx notation for vectors from Goodfellow
% https://github.com/goodfeli/dlbook_exercises

% alphabet templates
% abcdefghijklmnopqrstuvwxyz
% ABCDEFGHIJKLMNOPQRSTUVWXYZ

% fonts, math, and layout 
\usepackage{bbm}
\usepackage{enumerate}
\usepackage{amsmath}
\usepackage{amsthm}
\usepackage{amssymb}
\usepackage{amsfonts}
\usepackage{mathrsfs}
\usepackage{mathtools}
\usepackage[all]{xy}

% include graphics with \includegraphics
\usepackage{graphicx}
\usepackage{caption}

% \nicefrac{x}{y} gives a diagonal fraction bar x/y
\usepackage{nicefrac}

% definitions
\newtheorem{theorem}{Theorem}
\theoremstyle{definition}
\newtheorem{definition}{Definition}[section]

% url
\usepackage{hyperref}
\hypersetup{
    colorlinks=true,
    linkcolor=blue,
    filecolor=magenta,      
    urlcolor=cyan,
}

% brackets, norms, cardinalities
\newcommand{\pa}[1]{ \left({#1}\right) }
\newcommand{\ha}[1]{ \left[{#1}\right] }
\newcommand{\ca}[1]{ \left\{{#1}\right\} }
\newcommand{\inner}[1]{\left\langle #1 \right\rangle}
\newcommand{\innercpy}[1]{\inner{ #1, #1 }}
\newcommand{\norm}[1]{\left\lVert #1 \right\rVert}
\newcommand{\card}[1]{\left\lvert{#1}\right\rvert}
\newcommand{\abs}[1]{\card{#1}}

% math vectors
\newcommand{\va}{\mathbf{a}}
\newcommand{\vb}{\mathbf{b}}
\newcommand{\vc}{\mathbf{c}}
\newcommand{\vd}{\mathbf{d}}
\newcommand{\ve}{\mathbf{e}}
\newcommand{\vf}{\mathbf{f}}
\newcommand{\vg}{\mathbf{g}}
\newcommand{\vh}{\mathbf{h}}
\newcommand{\vi}{\mathbf{i}}
\newcommand{\vj}{\mathbf{j}}
\newcommand{\vk}{\mathbf{k}}
\newcommand{\vl}{\mathbf{l}}
\newcommand{\vm}{\mathbf{m}}
\newcommand{\vn}{\mathbf{n}}
\newcommand{\vo}{\mathbf{o}}
\newcommand{\vp}{\mathbf{p}}
\newcommand{\vq}{\mathbf{q}}
\newcommand{\vr}{\mathbf{r}}
\newcommand{\vs}{\mathbf{s}}
\newcommand{\vt}{\mathbf{t}}
\newcommand{\vu}{\mathbf{u}}
\newcommand{\vv}{\mathbf{v}}
\newcommand{\vw}{\mathbf{w}}
\newcommand{\vx}{\mathbf{x}}
\newcommand{\vy}{\mathbf{y}}
\newcommand{\vz}{\mathbf{z}}
\newcommand{\vzero}{\mathbf{0}}
\newcommand{\vone}{\mathbf{1}} 
\newcommand{\valpha}{{\boldsymbol\alpha}}
\newcommand{\vepsilon}{{\boldsymbol\epsilon}}
\newcommand{\veta}{{\boldsymbol\eta}}
\newcommand{\vsigma}{ {\boldsymbol\sigma}}
\newcommand{\vtheta}{ {\boldsymbol\theta}}
\newcommand{\vdelta}{ {\boldsymbol\delta}}
\newcommand{\vlambda}{ {\boldsymbol\lambda}}
\newcommand{\vmu}{ {\boldsymbol\mu}}
\newcommand{\vvartheta}{ {\boldsymbol\vartheta}}
\newcommand{\vbeta}{ {\boldsymbol\beta}}
\newcommand{\vphi}{ {\boldsymbol\phi}}


% math function arrows, misc binary math ops
\newcommand{\bij}{\leftrightarrow}
\newcommand{\inj}{\rightarrowtail}
\newcommand{\sur}{\twoheadedrightarrow}
\newcommand{\eqd}{\mathrel{\overset{\Delta}{=}}}

% common math sets
\newcommand{\Z}{\mathbb{Z}}
\newcommand{\R}{\mathbb{R}}
\newcommand{\C}{\mathbb{C}}
\newcommand{\N}{\mathbb{N}}
\newcommand{\Q}{\mathbb{Q}}
\newcommand{\F}{\mathbb{F}}
\newcommand{\T}{\mathbb{T}}

% limits
\def\sumn{\sum_{n=0}^\infty}
\def\limn{\lim_{n\rightarrow\infty}}
\def\prodn{\prod_{n=0}^\infty}

% mathcal
\newcommand{\mcA}{\mathcal{A}}
\newcommand{\mcB}{\mathcal{B}}
\newcommand{\mcC}{\mathcal{C}}
\newcommand{\mcD}{\mathcal{D}}
\newcommand{\mcE}{\mathcal{E}}
\newcommand{\mcF}{\mathcal{F}}
\newcommand{\mcG}{\mathcal{G}}
\newcommand{\mcH}{\mathcal{H}}
\newcommand{\mcI}{\mathcal{I}}
\newcommand{\mcJ}{\mathcal{J}}
\newcommand{\mcK}{\mathcal{K}}
\newcommand{\mcL}{\mathcal{L}}
\newcommand{\mcM}{\mathcal{M}}
\newcommand{\mcN}{\mathcal{N}}
\newcommand{\mcO}{\mathcal{O}}
\newcommand{\mcP}{\mathcal{P}}
\newcommand{\mcQ}{\mathcal{Q}}
\newcommand{\mcR}{\mathcal{R}}
\newcommand{\mcS}{\mathcal{S}}
\newcommand{\mcT}{\mathcal{T}}
\newcommand{\mcU}{\mathcal{U}}
\newcommand{\mcV}{\mathcal{V}}
\newcommand{\mcW}{\mathcal{W}}
\newcommand{\mcX}{\mathcal{X}}
\newcommand{\mcY}{\mathcal{Y}}
\newcommand{\mcZ}{\mathcal{Z}}

% measure theory, distributions
\newcommand{\indicator}{\mathbbm{1}}
\DeclareMathOperator{\Laplace}{Laplace}
\DeclareMathOperator{\soft}{soft}
\DeclareMathOperator{\hard}{hard}
\DeclareMathOperator{\vcdim}{vc}
\DeclareMathOperator{\clo}{clo}
\DeclareMathOperator{\Poisson}{Poisson}
\DeclareMathOperator{\PoissonBinomial}{PoissonBinomial}
\DeclareMathOperator{\Exponential}{Exponential}
\DeclareMathOperator{\Multinomial}{Multinomial}
\DeclareMathOperator{\Bernoulli}{Bernoulli}
\DeclareMathOperator{\Categorical}{Categorical}
\DeclareMathOperator{\Uniform}{Uniform}
\DeclareMathOperator{\Binomial}{Binomial}
\DeclareMathOperator{\Hypergeometric}{Hypergeometric}
\DeclareMathOperator{\GammaDist}{Gamma}
\DeclareMathOperator{\NegativeBinomial}{NegativeBinomial}
\DeclareMathOperator\mathProb{\mathbb{P}}
\DeclareMathOperator\sub{sub}
\renewcommand{\d}[1]{\mathop{\mathrm{d} #1 }}
\DeclarePairedDelimiterX{\infdivx}[2]{(}{)}{ #1\;\delimsize\|\;#2 }
\newcommand{\dkl}[2]{\mathop{D_\text{KL}}\infdivx{#1}{#2}}
\newcommand{\sg}{\mathop{\mathrm{SG}}}
\newcommand{\se}{\mathop{\mathrm{SE}}}

% distributions
\makeatletter
\newcommand{\distas}[1]{\mathbin{\overset{#1}{\kern\z@\sim}}}%
\makeatother
\newcommand{\disteq}{\overset{d}{=}}
\newcommand{\distiid}{\distas{\text{iid}}}
\newcommand\independent{\protect\mathpalette{\protect\independenT}{\perp}}
\def\independenT#1#2{\mathrel{\rlap{$#1#2$}\mkern2mu{#1#2}}}
\newcommand{\convdist}{\mathbin{\overset{\text{d}}{\longrightarrow}}}
\newcommand{\convarg}[1]{\mathbin{\overset{ #1 }{\longrightarrow}}}
\newcommand{\convas}{\mathbin{\overset{\text{as}}{\longrightarrow}}}
\newcommand{\convpb}{\mathbin{\overset{\text{pb}}{\longrightarrow}}}

\renewcommand{\P}{\mathProb} % need to overwrite stupid paragraph symbol
\DeclareMathOperator\mathExp{\mathbb{E}}
\DeclareMathOperator*\mathExpUnder{\mathbb{E}}
\newcommand{\E}{\mathExp}

\newcommand{\sa}{{\(\sigma\)-algebra}}
\newcommand{\OR}{{\overline{\R}}}
\newcommand{\OX}{{\overline{X}}}
\DeclareMathOperator{\power}{{\mathcal{P}}}
\DeclareMathOperator{\var}{var}
\DeclareMathOperator{\cov}{cov}

% \set{from set}{condition} with set-builder notation
% conditional expectation is analogous
\newcommand{\set}[2]{ \left\{ #1 \,\middle|\, #2 \right\} }
\newcommand{\CE}[2]{ \mathExp\left[ #1 \,\middle|\, #2 \right] }

% linear-algebra related
\newcommand{\rip}{\operatorname{RIP}}
\DeclareMathOperator{\svd}{svd}
\newcommand{\frob}[1]{\norm{#1}_{\mathrm{F}}}
\newcommand{\mat}[1]{\begin{pmatrix} #1 \end{pmatrix}}
\newcommand{\detmat}[1]{\begin{vmatrix} #1 \end{vmatrix}}
\DeclareMathOperator{\spanb}{span}
\DeclareMathOperator{\conv}{conv} % convex hull
\DeclareMathOperator{\cone}{cone}
\DeclareMathOperator{\vectorize}{vec}
\DeclareMathOperator{\matricize}{mat}
\DeclareMathOperator{\adj}{adj}
\DeclareMathOperator{\diag}{diag}
\DeclareMathOperator{\tr}{tr}
\DeclareMathOperator{\rank}{rank}
\DeclareMathOperator*{\argmin}{argmin}
\DeclareMathOperator*{\argmax}{argmax}
\DeclareMathOperator*{\proj}{proj}

% complex analysis
\DeclareMathOperator{\MyRe}{Re}
\DeclareMathOperator{\MyIm}{Im}
\DeclareMathOperator{\image}{image}
\DeclareMathOperator{\supp}{supp}

% typical numerical operators
\DeclareMathOperator{\sgn}{sgn}

% graphs
\DeclareMathOperator{\diam}{diam}

% constants
\renewcommand{\d}[1]{\mathop{\mathrm{d} #1 }}
\newcommand{\e}{\mathrm{e}}
\renewcommand{\i}{\imath}

% \bigtimes: large indexed cross product
\makeatletter
\DeclareFontFamily{U}  {MnSymbolF}{}
\DeclareSymbolFont{symbolsMN}{U}{MnSymbolF}{m}{n}
\SetSymbolFont{symbolsMN}{bold}{U}{MnSymbolF}{b}{n}
\DeclareFontShape{U}{MnSymbolF}{m}{n}{
    <-6>  MnSymbolF5
   <6-7>  MnSymbolF6
   <7-8>  MnSymbolF7
   <8-9>  MnSymbolF8
   <9-10> MnSymbolF9
  <10-12> MnSymbolF10
  <12->   MnSymbolF12}{}
\DeclareFontShape{U}{MnSymbolF}{b}{n}{
    <-6>  MnSymbolF-Bold5
   <6-7>  MnSymbolF-Bold6
   <7-8>  MnSymbolF-Bold7
   <8-9>  MnSymbolF-Bold8
   <9-10> MnSymbolF-Bold9
  <10-12> MnSymbolF-Bold10
  <12->   MnSymbolF-Bold12}{}
\DeclareMathSymbol{\tbigtimes}{\mathop}{symbolsMN}{2}
\newcommand*{\bigtimes}{%
  \DOTSB
  \tbigtimes
  \slimits@ 
}
\makeatother

% category theory arguments
% See https://tex.stackexchange.com/questions/356873
\newcommand{\catfst}{{-}}
\newcommand{\catsnd}{{=}}
\newcommand{\cattrd}{{\equiv}}
\DeclareMathOperator{\Id}{Id}
\newcommand{\opcat}[1]{{#1}^{\text{op}}}
\DeclareMathOperator{\Hom}{hom}
\DeclareMathOperator{\Ob}{ob}
\DeclareMathOperator{\El}{el}
\DeclareMathOperator{\colim}{colim}
\DeclareMathOperator{\Sym}{Sym}
% restriction of a function to a domain
\newcommand\restr[2]{{% we make the whole thing an ordinary symbol
  \left.\kern-\nulldelimiterspace % automatically resize the bar with \right
  #1 % the function
  \vphantom{\big|} % pretend it's a little taller at normal size
  \right|_{#2} % this is the delimiter
}}

\newtheorem{lemma}{Lemma}


\begin{document}

\maketitle

\section{The 1-dimensional It\^{o} Formula}

\begin{definition}[It\^{o} Process]
  Consider the probability space \((\Omega, \mcH,\P)\) where \(\mcH=\mcH_\infty\) for a filtration \(\ca{\mcH_t}\) for which the Brownian motion \(B_t\) is adapted. Let \(u\) be an \(H_t\) adapted as well, and \(L^1([0,T])\) almost surely in its first argument, and \(v\in\mcW_\mcH\), where \(\mcW_\mcH\) is defined in the previous chapter as a relaxation of \(\mcV_\mcH\) to almost surely \(L^2([0,T])\) processes for all \(T\ge 0\).

  Then an It\^{o} process is a process \(X_t\) satisfying
  \[
    \d{X_t}=u \d{t}+v\d{B_t}\,\,,
  \]
  which is shorthand for \(X_T(\omega)=X_0+\int_0^T u(t,\omega)\d{t}+\int_0^T v(t, \omega)\d{B_t}\) holding almost surely for all \(T\ge 0\).
\end{definition}

In certain cases, we can find the functional form of an It\^{o} process through the analogue of the chain rule.

\begin{theorem}[The 1-dimensional It\^{o} Formula]
  Let \(X_t\) be an It\^{o} process satisfying \(\d{X_t}=u \d{t}+v\d{B_t}\) and suppose \(g\in\mcC^2(\R_+\times\R)\). Then \(Y_t=g(t, X_t)\) is an It\^{o} process satisfying
  \[
   \d{Y_t}=\partial_1g(t, X_t)\d{t}+\partial_2g(t, X_t)\d{X_t}+\frac{1}{2}\partial_2^2g(t, X_t)\pa{\d{X_t}}^2 \,\,,
  \]
  where \((\d{X_t})^2\) is provocative sugar for \(\d{t}\).
\end{theorem}
\begin{proof}
  Letting \(V_s=v(s, \omega)\) and \(U_s=u(s, \omega)\) and desugaring, we must verify that \(g(t, X_t)-g(0, X_0)\) equals
  \[
   I(g, t, X)=\int_0^t\pa{\partial_1g(s,X_s)+U_s\partial_2g(s, X_s)+\frac{1}{2}V_s^2\partial_2^2g(s, X_s)}\d{s}+\int_0^tV_s\partial_2g(s,X_s)\d{B_s}\,\,.
  \]

  Before jumping into verification, we must first simplify the problem to work with bounded \(g\) and elementary \(u,v\). To argue without loss of generality, we first perform our reductions.

  \textbf{It suffices to consider bounded \(g\).} Define the stopping time \(\tau_n=\inf\set{s>0}{\abs{X_s}\ge n}\). Define \(g_n\) to equal to \(g\), but be supported on only \([0,n]\times[-n,n]\) and \(\partial_1g,\partial_2^2g\); by continuity and compact support each \(g_n\) and its derivatives are bounded.

  Defining \(\int_0^{t\land \tau_n}\cdots\) to be \(\int_0^{t}1\ca{s\le \tau_n}\cdots\), it's clear that \(I(g_n, t, X)=I(g,t\land \tau_n, X)\). For fixed \(t\), as \(X_t\) is \(L^2(\P\times [0,t])\)-integrable, \(\P\ca{\tau_n>t}\rightarrow 1\), which implies that \(I(g_n,t, X)=I(g, t\land \tau_n, X)\convas I(g, t, X)\). Since \(g_n\convas g\) as well, we can lift a proof of the It\^{o} formula from bounded \(g\) to unbounded.

\textbf{It suffices to consider elementary \(u,v\)}.
Furthermore, we may assume \(u,v\) are elementary functions; by integrability and adaptedness (and hence by Monotone Class Theorem) for \(u\) and It\^{o}'s construction elementary \(u_n,v_n\) exist with \(\d{X_t^{(n)}}=u_n\d{t}+v_n\d{B_t}\) for which \(X_t^{(n)}\convarg{L^1(\P)} X_t\), implying almost sure convergence along some subsequence. Then by continuous mapping theorem we have \(g(t, X_t^{(n)})\convas g(t, X_t)\). By boundedness of \(g\) each integral integrand of \(I(g, t, X^{(n)})\) is itself bounded, so by BCT \(I(g, t, X^{(n)})\convarg{L^1(\P)}I(g, t, X)\), and therefore through \(I(g, t, X^{(n)})=g(t, X_t^{(n)})\) we have almost sure equality of the limits.

  \textbf{With these amenities out of the way}, by application of Taylor's theorem at each \(g_j=g(t_j,X_{t_j})\), the difference \(g(t, X_t)-g(0, X_0)\) is given by

  \[
    \sum_j\Delta_jg=\sum_j\partial_1g_j\Delta_jt+\partial_2g_j\Delta_jX+\frac{1}{2}\partial_1^2 g_j\pa{\Delta_jt}^2+\frac{1}{2}\partial_2^2g_j\pa{\Delta_jX}^2+\partial_1\partial_2g_j\Delta_jt\Delta_jX+R_j\,\,,
  \]
  with the remainder \(R_j=o\pa{\pa{\Delta_jt}^2+\pa{\Delta_jX}^2}\).

  Under the limit of shrinking partitions, \(\sum_j\partial_1g_j\Delta_jt\rightarrow\int_0^t\partial_1g(s,X_s)\d{s}\), holding almost surely and then by a limit interchange (relying on \(g\)'s boundedness) in \(L^1(\P)\) as well.

  Similarly, we have \(\sum_j\partial_2g_j\Delta_jX\rightarrow\int_0^t\partial_2g(s,X_s)\d{B_s}\) in \(L^2(\P)\), since \(g\in \mcV(0,t)\). Next, we leverage the fact that \(u,v\) are elementary such that for sufficiently fine grids, there exist constants \(u_j,v_j\) such that
  \[
    \pa{\Delta_j X}^2=u_j^2\pa{\Delta_jt}^2+2u_jv_j\Delta_jt\Delta_jX+v_j^2\pa{\Delta_jB}^2\,\,.
  \]

  In the original sum, for some bounded \(c_j\), we find that terms \(\sum_jc_j\Delta_jt\Delta_jX,\sum_jc_j\pa{\Delta_jt}^2\) vanish in \(L^2(\P)\), the former's second moment being \(O(n^{-3})\) and the latter's being \(O(n^{-4})\).

  This leaves the conspicuous term \(\frac{1}{2}\sum_j\partial_2^2g_j\pa{\Delta_jB}^2\), which we may show tends to \(\int_0^t\partial_2^2g(s, X_s)V_s^2\d{s}\) in \(L^2(\P)\) by triangle inequality and comparing to the analogous sequence with \(\Delta_jt\) instead. A calculation involving normal moments and leveraging independent increments finds that indeed
  \[\E\ha{\pa{\frac{1}{2}\sum_j\partial_2^2g_j\pa{\pa{\Delta_jB}^2-\Delta_jt}}^2}\rightarrow 0\,\,.\]

  Since every term converges to the desired parts of our integral sum in \(L^2(\P)\) for some sequence, certainly we can choose a subsequence that converges almost surely. Since the discrete differences always equal \(g(t, X_t)-g(0, X_0)=\sum_j\Delta_jg\), the It\^{o} formula equality holds in the limit.
    \end{proof}

    \begin{theorem}[Stochastic Integration by Parts]
      If \(f(s,\omega)=f(s)\) is deterministic and of bounded variation in \([0,t]\) such that the \href{https://en.wikipedia.org/wiki/Riemann%E2%80%93Stieltjes_integral}{Riemann-Stieltjes integral} is well-defined, then
        \[
          \int_0^t f(s)\d{B_s}=f(t)B_t-\int_0^tB_s\d{f_s}
          \]
        \end{theorem}
        \begin{proof}
          In the case of \(f\in\mcC^2\), this follows from the It\^{o} formula. But for the more general class, a direct proof is needed. For a sequence of shrinking partitions of \([0, t]\),
          \[
            f(t)B_t=\sum_j\Delta_j(fB)=\sum_jB_{t_{j+1}}\Delta_j f+\sum_jf_{t_j}\Delta_j B\convas \int_0^tB_s\d{f_s}+\int_0^t f(s)\d{B_s}\,\,,
          \]
          which is apparent after adding and subtracting the term \(f_{t_j}B_{t_{j+1}}\) within \(\Delta_j(fB)=f_{t_{j+1}}B_{t_{j+1}}-f_{t_{j}}B_{t_{j}}\).
          \end{proof}
          
\section{The Multi-dimensional It\^{o} Formula}

For vector-valued \(u:(\R_+,\R^n)\rightarrow\R^n\) and matrix-valued \(v:(\R_+,\R^n)\rightarrow\R^{m\times n}\), such that \(\d{X_t}=u\d{t}+v\d{B_t^{(m)}}\) has well-defined It\^{o} process components, for a conformal \(g\in\mcC^2\), an analogous multi-dimensional It\^{o} formula may be derived in the same manner as above via Taylor's theorem.

\section{The Martinagle Representation Theorem}

Note that this theorem has a similar multi-dimensional extension.

\begin{theorem}[Martingale Representation Theorem]
  Suppose \(M_t\) is a square-integrable martinagle with respect to the filtration generated by Brownian motion. Then there exists a stochastic process \(g(s,\omega)\in \mcV(0, t)\) such that
  \[
    M_t=\E\ha{M_0}+\int_0^tg(s,\cdot)\d{B_s}\,\,,
  \]
  holding almost surely for all \(t\ge 0\). 
\end{theorem}
\begin{proof}
  The proof proceeds in several steps. First; for any \(T>0\) the set random variables defined by evaluating analytic functions with compact support on \(R^n\) on any finite tuple of Brownian motions \((B_{t_1},\cdots,B_{t_n})\) is dense in \(L^2(\mcF_T,\P)\), the set of real-valued square-integrable rvs measurable with respect to \(\mcF_T\).

  Second; the linear span of random variables of type \(\exp\pa{\int_0^Th(t)\d{B_t}-\frac{1}{2}\int_0^Th^2(t)\d{t}}\) for all (deterministic) \(h\in L^2([0,T])\) is also dense in \(L^2(\mcF_T, \P)\) by virtue of spanning the set described above over analytic functions with compact support.

  By the density of the class described above, it suffices to consider only random variables of the aforementioned form. For any such \(Y_T\) defined by some \(h\), as the final representation equality holds under linear transformation. Since \(Y_t=\exp(X_t)\), where \(\d{X_t}=-\frac{1}{2}h^2(s)\d{s}+h(s)\d{B_s}\). Applying It\^{o}'s formula yields an equality exactly of the desired form \(Y_T=1 + \int_0^tg(s,\cdot)\d{B_s}\). In turn, for any square-integrable, \(\mcF_T\)-measurable random variable, such as \(M_T\), by linearity we may write it as \(\E[M_T]+\int_0^Tg_{T}(s,\cdot)\d{B_s}\).

  This holds for all \(T\), and \(\E\ha{M_T}=\E \ha{M_0}\) uniformly, but we must show that the \(g_T\) values are the same across \(T\) values. Luckily, this holds due to Exercise 15 of Chapter 3.
  \end{proof}


  \section{Exercises}

  \subsection{}

  In the following, we choose a base process and \(\mcC^2(\R_+\times \R)\) function \(g\) to derive a canonical It\^{o} form for a given stochastic process.

  \subsubsection{}
  \(X_t=B_t^2\), choose \(g(t, x)=x^2\) so \(X_t=g(t, B_t)\). Then,
  \[\d{X_t}=\d{t}+2B_t\d{B_t}\,\,.\]

  \subsubsection{}
  \(X_t=2+t+\exp\pa{B_t}\), choose \(g(t, x)=2+t+\exp(x)\) so \(X_t=g(t, B_t)\). Then,
  \[\d{X_t}=\pa{1+\frac{1}{2}\exp(B_t)}\d{t}+\exp\pa{B_t}\d{B_t}\,\,.\]
  \subsubsection{}
  \(X_t=B_1(t)^2+B_2(t)^2\), choose \(g\pa{t, \mat{x_1\\x_2}}=x_1^2+x_2^2\) so \(X_t=g(t, B_t^{(2)})\). Then,
  \[\d{X_t}=2\d{t}+2B_1(t)\d{B_1(t)}+2B_2\d{B_2(t)}\,\,.\]
  \subsubsection{}
  \(X_t=\mat{t_0+t& B_t}^\top\), choose \(g(t, x)=\mat{t_0+t&x}^\top\) so \(X_t=g(t, B_t)\). Then,
  \begin{align*}
    \d{X_1(t)}&=\d{t}\\
    \d{X_2(t)}&=\d{B_t}\,\,.
  \end{align*}

  \subsubsection{}
  \(X_t=g(t, B_1(t), B_2(t), B_3(t)) = \mat{B_1(t)+B_2(t)+B_3(t) & B_2^2(t)-B_1(t)B_3(t)}\). Then,
  \begin{align*}
    \d{X_1(t)}&=\d{B_1(t)}+\d{B_2(t)}+\d{B_3(t)}\\
    \d{X_2(t)}&=\d{t}-B_3(t)\d{B_1(t)}+2B_2(t)\d{B_2(t)}-B_1(t)\d{B_3(t)}\,\,.
  \end{align*}

  \subsection{}
  % 4.2
  Letting \(X_t=g(t, B_t)=\frac{1}{3}B_t^3\), It\^{o}'s formula gives us
  \[
\d{X_t}=B_t\d{t}+B_t^3\d{B_t}\,\,,
\]
from which we can derive the desired fact that
\[
  \int_0^tB_s\d{B_s}=\frac{1}{3}B_t^3-\int_0^tB_s\d{s}\,\,.
\]

\subsection{}
% 4.3
\label{ex:ibp}
With \(X_t,Y_t\) It\^{o} processes, letting \(Z_t=g(t, X_t, Y_t)=X_tY_t\) and applying It\^{o}'s formula yields
\[
  \d{Z_t}=Y_t\d{X_t}+X_t\d{Y_t}+\d{X_t}\cdot\d{Y_t}\,\,,
\]
which by de-sugaring yeilds integration by parts:
\[
  \int_0^tX_s\d{Y_s}=X_tY_t-X_0Y_0-\int_0^tY_s\d{X_s}-\int_0^t\d{X_s}\cdot\d{Y_s}\,\,,
\]
where by virtue of being It\^{o} processes, square integrability, adaptedness immediately follow. Notably, the joint measurability condition, requiring that \((t,\omega)\mapsto X_t(\omega)\) is \((\mcB(\R_+)\times \mcF_\infty)\)-measurable, follows by chosing an appropriate version of \(X_t\), which is proven to exist in Chapter 3.

\subsection{} % 4.4

Let \(\theta(t)\in\mcV^n(0, T)\). Define
\[
  Z_t=\exp\pa{\int_0^t\theta(s)\cdot\d{B(s)}-\frac{1}{2}\int_0^t\|\theta(s)\|_2^2\d{s}}\,\,,
\]
where \(B(t)\) is Brownian motion in \(\R^n\). We will canonicalize \(Z_t\) as an It\^{o} process and show a martinagle property.

Let \(Y_t\) be an rv such that \(\d{Y_t}=\theta(s)\cdot\d{B(t)}-\frac{1}{2}\|\theta(s)\|_2^2\d{t}\). Notice this is well-formed, with measurability and adaptiveness properties following from \(\theta(t)\in\mcV^n\), with the only nuanced point being that \(\|\theta(s)\|_2^2\in L^1[0,T]\) holding almost surely by virtue of each component of \(\theta(s)\) being in \(L^2(\P\times[0,T])\).

Then \(Z_t=g(t, Y_t)=\exp Y_t\) with \(Y_0=0\). An application of a 1-dimensional It\^{o}'s formula, albeit with a wider filtration \(\mcH_t=\mcF_t^{(n)}\), yields
\[
  \d{Z_t}=\exp Y_t \theta(t)\cdot \d{B(t)}= Z_t \theta(t)\cdot \d{B(t)}\,\,,
\]
where by the martingale property of smooth It\^{o} processes, \(Z_t\) is a martingale if \(Z_t\theta(t)\) components are in \(mcV\). This is because every consitutent \(Z_t\theta_k(t)\d{B_k}\) is itself a martingale with respect to \(\mcH\), so the sum of such terms remains one.

\subsection{} % 4.5

Let \(B_t\) be 1-dimensional Brownian motion with \(B_0=0\). With It\^{o}'s formula we can derive a simple recurrence for even moments of the Normal distribution.

In particular, letting \(g_k(t, x)=x^k\), It\^{o}'s formula applied to \(g_k(t, B_t)\) yeilds
\[
  \d{B_t^k}=\frac{1}{2}k(k-1)B_t^{k-2}\d{t}+kB_t^{k-1}\d{B_t}\,\,,
\]
holding for all \(k\) almost surely. Then applying expectations to both sides, and defining a notation for the moments \(\beta_k(t)=\E B_t^k\), by virtue of It\^{o} integrals being mean-zero, yields the recurrence relation
\[
  \beta_k(t)=\frac{1}{2}k(k-1)\E \int_0^tB_t^{k-2}\d{t}=\frac{1}{2}k(k-1)\int_0^t\beta_{k-2}(t)\d{t}\,\,,
\]
by an application of DCT. In turn, we are able to do some computation to find that \(\beta_2(t)=t\), \(\beta_4(t)=3t^2\), and \(\beta_6(t)=15t^3\).

\subsection{} %4.6

Let \(c,\alpha\) be constants. Then setting \(X_t=g(t, B(t))=\exp\pa{ct+\sum_{j=1}^n\alpha_jB_j(t)}\) for \(n\)-dimensional Brownian motion \(B(t)\) and applying It\^{o}'s formula yields
\[
  \d{X_t}=\pa{c+\frac{1}{2}\sum_{j=1}^n\alpha_j^2}X_t\d{t}+X_t\sum_{j=1}^{n}\alpha_j\d{B_j}\,\,,
\]
which follows from computation without much fanfare (except the substitution for the definition of \(X_t\) at the end).

\subsection{} % 4.7

Let \(X(t)\) be defined by \(\d{X(t)}=v\times \d{B(t)}\) where \((t,\omega)\mapsto v_t(\omega)\in\mcV^n\) with \(X\in\R^n,B(t)\in\R^n\).

Note that \(\|X(t)\|_2^2\) is in general not a martingale; a simple example is \(v=1\), where \(X_t=B_t\) so its square norm is \(t\) in expectation, which is not constant-mean and thus not a martingale.

On the other hand, define \(M_t=\|X(t)\|_2^2-\int_0^t\|v(s)\|_2^2\d{s}\). We verify that this is indeed a martingale.

Integrability and adaptedness follow by inspecting each component of \(M_t\). To show the martingale property, that \(\CE{M_t}{\mcF_s^{(n)}}=M_s\), we require two observations. First, the equality is completely symmetric in each component of \(X(t)\), so it suffices to show this in just one dimension. Second, by expanding \(X_t=(X_t-X_s)+X_s\), \(\CE{M_t}{\mcF_s^{(n)}}=M_s+\CE{(X_t-X_s)^2}{\mcF_s^{(n)}}-\int_s^tv_s^2\d{s}\).

We conclude by observing that \(\CE{(X_t-X_s)^2}{\mcF_s^{(n)}}=\int_s^tv_s^2\d{s}\) by an application of It\^{o}'s isometry. This requires \(X_t\in\mcV\) and for \(X_t\) to be a martingale (which it is), the latter promoting \(\CE{(X_t-X_s)^2}{\mcF_s^{(n)}}\) to an unconditional \(\E{(X_t-X_s)^2}\).

  \textbf{Observation}. There doesn't seem to be a need for boundedness of \(v\), which is curious.

\subsection{} % 4.8

Let \(\laplace=\sum_i\partial_i^2\) be the Laplace operator. Then It\^{o}'s formula, applied to \(f(B(t))\), reduces to
\[
  f(B(t))-f(B(0))=\int_0^t\nabla f(B_s)\cdot \d{B_s}+\frac{1}{2}\int_0^t\laplace f(B_s)\d{s}\,\,.
  \]

  Consider now a \(g\in \mcC^1\) which is \(\mcC^2\) everywhere but a finite set of points, with \(\abs{g''}\le M\) where defined. Then the above still holds.

  Indeed, similar to the construction of the It\^{o} integral for non-continuous functions, consider a convolution \(g_k''=\phi_k*g''\) for boxcars \(\phi_k\) with shrinking support in \(k\). This is well-defined everywhere on the domain since the convolution is unique up to measure-zero sets. Along with initial conditions, this induces a set of functions \(g_k',g_k\) by integration, all of which have the property that \(g_k\rightarrow g,g_k'\rightarrow g'\) uniformly (since \(g'',g_k''\) are bounded). We also have \(g_k''\rightarrow g''\) outside the specified measure-zero set. By boundedness of \(g_k''\) and \(M\)-Lipschitzness \(g_k'\), we can exchange the integrals and limits of the above formula; so the mere pointwise convergence of \(g_k,g_k',g_k''\) is sufficient.
\label{ex:discont}
  \subsection{} % 4.9
  
  Exercise 4.9 is completed inline in the proof of It\^{o}'s formula provided above.

  \subsection{} % 4.10

  We can't directly apply It\^{o}'s formula to \(g(B_t)\), where \(g(x)=\abs{x}\) is \(\mcC^2\) everywhere with uniformly bounded second derivative except the origin. Instead, consider the Huber approximation
  \[
    g_\epsilon(x)=\begin{cases}
      \abs{x}& \abs{x}\ge \epsilon\\
      \frac{1}{2}\pa{\epsilon + \frac{x^2}{\epsilon}} & \text{o/w}
      \end{cases}\,\,.
    \]
    which is \(\mcC^2\) everywhere but at \(\pm\epsilon\) with jump discontinuities, but still uniformly bounded second derivatives.

    By the previous Exercise~\ref{ex:discont},
    \begin{align*}
      g_\epsilon(B_t)-g_\epsilon(B_0)&=\int_0^tg_\epsilon'(B_s)\d{B_s}+\frac{1}{2}\int_0^tg_\epsilon''(B_s)\d{s}\\
      &=\int_0^tg_\epsilon'(B_s)\d{B_s}+\frac{1}{2}\int_0^t1\ca{\abs{B_s}\le\epsilon}\d{s}\,\,,
    \end{align*}
    despite the jump discontinuities of the second derivative. Let
    \[
      L_t(\epsilon)=\frac{1}{2}\int_0^t1\ca{\abs{B_s}\le\epsilon}\d{s}=\frac{1}{2\epsilon}\lambda\pa{\set{s\in[0,t]}{\abs{B_s}\le \epsilon}}\,\,,
    \]
    where \(\lambda\) is the Lebesgue measure. We now turn our attention to the remaining integral term
    \[
      \int_0^tg_\epsilon'(B_s)\d{B_s}=\underbrace{\int_0^tg_\epsilon'(B_s)1\ca{\abs{B_s}\le\epsilon}\d{B_s}}_{A_\epsilon}+\underbrace{\int_0^tg_\epsilon'(B_s)1\ca{\abs{B_s}>\epsilon}\d{B_s}}_{B_\epsilon}\,\,.
    \]

    First, we show \(A_\epsilon\rightarrow 0\) in \(L^2(\P)\) as \(\epsilon\rightarrow 0\).
    \begin{align*}
      \E\ha{A_\epsilon^2}
      &=\E\int_0^t\pa{g_\epsilon'(B_s)1\ca{\abs{B_s}\le \epsilon}}^2\d{s} & \text{It\^{o} isometry}\\
      &=\frac{1}{\epsilon^2}\E\int_0^tB_s^21\ca{\abs{B_s}\le \epsilon}\d{s} & \text{def.~of } g_\epsilon\\
      &=\frac{1}{\epsilon^2}\int_0^t\CE{B_s^2}{\abs{B_s}\le \epsilon}\P\ca{\abs{B_s}\le \epsilon}\d{s} & \text{DCT}\\ 
      &\le \int_0^t\P\ca{\abs{B_s}\le \epsilon}\d{s} & \CE{B_s^2}{\abs{B_s}\le \epsilon}\le\epsilon^2\\
      &\le O(\epsilon)
    \end{align*}

    To finish, notice that the \(N(0,s)\) density is maximized at the origin by \((2\pi s)^{-1/2}\). Then \(\P\ca{\abs{B_s}\le \epsilon}=O\pa{\epsilon s^{-1/2}}\). The integral \(\int_0^ts^{-1/2}\d{s}=\int_{t^{-1/2}}^\infty u^{-2}\d{u}=\sqrt{t}\) by reflection across unity which is \(O(1)\) relative to \(\epsilon\).

    Finally, by inspection of \(B_\epsilon\) and letting \(\epsilon\rightarrow 0\), we have
    \[
      \abs{B_t}-\abs{B_0}=\int_0^t\sgn(B_s)\d{B_s}+L_t\,\,,
    \]
    where \(L_t=\lim_{\epsilon\rightarrow 0}L_t(\epsilon)\). Such a limit surely exists in \(L^2(\P)\)---the same integral we just performed would prove it!.

    \subsection{} % 4.11

    \begin{lemma} Let \(X_t=g(t, B_t)\) with \(g\in\mcC^2\). Then by It\^{o}'s formula, if \(\partial_1g(t, B_t)+\frac{1}{2}\partial_2^2g(t, B_t)=0\), \(X_t\) is an It\^{o} integral. If, further, \(\partial_2 g(t, B_t)\in \mcV\) (integrability in \(L^2(\P\times [0,T])\) being the only condition not implied by \(g\in\mcC^2\) here), then \(X_t\) must be a martingale by properties of the It\^{o} integral.
    \end{lemma}

    By computation, the lemma above verifies that the following are martingales with respect to the natural Brownian motion filtration:

    \begin{enumerate}
    \item \(X_t=\exp\pa{\frac{1}{2}t}\cos B_t\)
    \item \(X_t=\exp\pa{\frac{1}{2}t}\sin B_t\)
    \item \(X_t=(B_t+t)\exp\pa{-B_t-\frac{1}{2}t}\).
    \end{enumerate}

    \subsection{} % 4.12
    Let \(\d{X(t)}=u(t)\d{t}+v(t)\times \d{B(t)}\) with \(u,v\) stochastic processes in \(\R^n\) such that the It\^{o} process \(X(t)\) is well defined. Further, assume that
    \[
      \E\int_0^t\norm{u(s)}\d{s}+      \E\int_0^t\norm{v(s)v^\top(s)}\d{s}<\infty\,\,,
    \]
    and that \(X_t\) is a martingale with respect to \(\mcF\). Then we will show that \(u=\textbf{0}\) almost everywhere on \(\R_+\times\Omega\).

    Since \(X_t\) is a martingale, it must be that \(\CE{X(t)}{\mcF_s}=X(s)\) for all \(t> s\). Subtracting out the \(\mcF_s\)-measurable components of the integrals \(\int_0^su(r)\d{r}\) and \(\int_0^sv(r)\times \d{B_r}\) from both sides, which yields
    \[
\CE{\int_s^tu(r)\d{r}}{\mcF_s}+\CE{\int_s^tv(r)\times\d{B(r)}}{\mcF_s}=\textbf{0}\,\,.
      \]
      In a lemma after this proof, we will show that \(\CE{\int_s^tv_r\times\d{B_r}}{\mcF_s}=\textbf{0}\) by a general property of It\^{o} integrals. As a result, \(\CE{\int_s^tu(r)\d{r}}{\mcF_s}=\textbf{0}\). Appealing to DCT and applying Liebniz's rule,
      \[
        \partial_t\CE{\int_s^tu(r)\d{r}}{\mcF_s}=\CE{\partial_t\int_s^tu(r)\d{r}}{\mcF_s}=\CE{u(t)}{\mcF_s}\,\,,
      \]
      implying that the latter is \(\textbf{0}\) as well. Then taking the limit \(s\rightarrow t^{-}\) and applying Corollary C.9 we find that \(\CE{u(t)}{\mcF_{\lim}}=u(t)=\textbf{0}\) where \(\mcF_{\lim}=\sigma\ca{\mcF_s}_{s<t}=\mcF_t\), the latter fact holding by choosing a {\em surely} continuous (and therefore left-continuous) version of \(B_t\) such that its generated natural sigma algebra is {\em surely} predicted from the filtrations leading up to it. Event wise, by left continuity of the generating process, for any \(A\in\mcF_t\) there's a countable monotonic sequence of events \(A_s\nearrow A\) as \(s\rightarrow t\) with each \(A_s\in\mcF_s\), so that \(\lim_s A_s\in \mcF_{\lim}\) by the limit closure property of a sigma algebra.

      From this argumentation, it's clear why the filtration of the martingale matters; the lke

      \begin{lemma}[Mean-zero Conditional Property for It\^{o} Integrals]
        Let \(f\in\mcV(S, T)\). Then \(\CE{\int_S^Tf_t\d{B_t}}{\mcF_s}=0\) almost surely. Note that this is not quite as strong as independent increments.
      \end{lemma}
      \begin{proof}
        Directly from It\^{o}'s construction, there exist some elementary \(\phi_n\convarg{L^2(\P\times[S,T])}f\) for whom
        \[\int_S^T\phi_n(t)\d{B_t}\convarg{L^2(\P)}\int_S^Tf_t\d{B_t}\,\,.
        \]
        Inspecting such an elementary integral, we find that
        \[
          \int_S^T\phi_n(t)\d{B_t}=\sum_je_j^{(n)}\Delta_j B\,\,,
        \]
        for some finite set of fixed rvs \(e_j^{(n)}\in\mcF_{t_j}\) and increments \(\Delta_j B\) from a partition of \([S, T]\). But taking \(\CE{\cdot}{\mcF_s}\) of both sides here shows that the elementary integral must be zero since it is the sum of finitely many scaled independent increments \(\Delta_jB\) from the future, requiring an application of the tower property:
        \[
          \CE{\sum_je_j^{(n)}\Delta_j B}{\mcF_s}=\CE{\sum_j\CE{e_j^{(n)}\Delta_j B}{\mcF_{t_j}}}{\mcF_s}=\CE{\sum_je_j^{(n)}\CE{\Delta_j B}{\mcF_{t_j}}}{\mcF_s}=0\,\,.
          \]
        In turn, as \(\int_S^T\phi_n(t)\d{B_t}\convas\int_S^T f_t\d{B_t}\) as well, by uniqueness of It\^{o} integrals, the limit holds after applying conditional expectations \(\CE{\cdot}{\mcF_s}\), which shows the desired lemma.
      \end{proof}
        \subsection{} % 4.13
        Let \(\d{X_t}=u_t\d{t}+\d{B_t}\). Assume \(u\in\mcV\). We will show that \(Y_t=X_t\exp \pa{-Z_t}\), where \(\d{Z_t}=\frac{u_t^2}{2}\d{t}+u_t\d{B_t}\), is a martingale.

        By It\^{o}'s formula applied to \(g(t, x, z)=x\exp\pa{-z}\), we find that \(\d{Y_t}=\d{g(t, X_t, Z_t)}\) has no \(\d{t}\) term, and must then be a martingale.
        \subsection{} % 4.14
        For each of the following, we find \(f\in\mcV(0, T)\) such that for the given \(F\), we have the almost sure equality
        \[
          F(\omega)=\E\ha{ F} + \int_0^T f_t(\omega)\d{B_t(\omega)}\,\,.
        \]

        We will do so by appropriate choice of \(g\) such that we can define a process \(X_t=g(t, B_t)\) with the property that \(X_T=F\), \(X_0=\E F\), and \(\d{X_t}=f_t\d{B_t}\) for some \(f_t\) to be derived since we chose \(g\) such that \(2\partial_1g+\partial_2^2g=0\), so there's no \(\d{t}\) term.

        \subsubsection{}

        \(F=B_T\). Choose \(g(t, x)=x\), yielding \(f_t=1\).
        \subsubsection{}

        \(F=\int_0^TB_t\d{t}\). Here, we use Exercise~\ref{ex:ibp} with parts \(X_t=B_t,Y_t=t\), yielding
        \[
          \int_0^TB_t\d{t}=B_TT-\int_0^Tt\d{B_t}
        \]
        from integration by parts of \(\d{\pa{X_tY_t}}\). Since \(TB_T=\int_0^T T\d{B_t}\), \(f_t=T-t\) suffices.
        \subsubsection{}

        \(F=B_T^2\). Choose \(g(t, x)=x^2-t\), yielding \(f_t=2B_t\).
        \subsubsection{}

        \(F=B_T^3\). Temporarily, define \(F'=B_T^3-3TB_T\). Choosing \(g'(t, x)=x^3-3tx\), yields \(f_t'=3B_t^2-3t\).b

        Then notice that \(F = F'+3TB_T=\E F' + \int_0^T\pa{f_t'+3T}\d{B_t}\), so set \(f_t=f_t'+3T\) to finish as \(\E F'=\E F\).
        
        \subsubsection{}

        \(F=\exp B_T\). Choose \(g(t, x)=\exp\pa{x-\frac{t}{2}+\frac{T}{2}}\), yielding \(f_t=g(t, B_t)\). This \(g\) was derived in the same manner as the previous exercise.
        \subsubsection{}

        \(F=\sin B_T\). Choose \(g(t, x)=\exp\pa{\frac{t-T}{2}}\sin x\), yielding \(f_t=\exp\pa{\frac{t-T}{2}}\cos B_t\). This was derived as before, but required performing the same ``\(F'\) iteration'' to a fixed point.
        
    \subsection{} % 4.15

    The desired result is a direct computation of It\^{o}'s formula.
\end{document}